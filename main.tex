\documentclass[10pt,twocolumn]{article} 

% use the oxycomps style file
\usepackage{oxycomps}

% read references.bib for the bibtex data
\bibliography{references}

% include metadata in the generated pdf file
\pdfinfo{
    /Title (Tarot Web App: Ethical Considerations)
    /Author (Darius Amir)
}

% set the title and author information
\title{Tarot Web App: \\Ethical Considerations}
\author{Darius Amir}
\affiliation{Occidental College}
\email{mamir.edu}

\begin{document}

\maketitle

\section{Project Description and Introduction}
    The proposed project aims to create a tarot web app. This will be done through the form of a web page that is formatted like an adventure dialogue. Users will be met with a visually pleasing character once they visit a site and be presented with a quick introduction to the functions available. They will then be able to choose a specific reading topic and pick however many cards they wish from 1 to a set number. They would make their choice by pressing keys on a default keyboard. Afterwards, they will be given a reading that is generated using a machine learning LSTM algorithm. The app would be tested on the developer's Instagram and/or Tik Tok followers, who would be compensated by personalized readings done by the developer himself. 
    
    However, there are many ethical issues that may and will arise in this project. They include, but are not limited to: accessibility issues, shifts in the tarot reading market, users overly depending on and/or believing false tarot readings, many user testing biases, consent issues if a user asks for a reading about someone who didn’t consent to it, and a misunderstanding when it comes to the origin of generated readings. As a result of these numerous shortcomings and damages the app may cause, it should not be created. 
    
    The next sections will visit these specific ethical issues. The final section will propose some possible solutions and highlight that the app would still be harmful despite these solutions going into effect.


\section{Accessibility}
    The content of the site is said to be delivered through written text. This would pose major issues for those who are blind or face lower vision. While many of these affected individuals do employ the use of screen readers\cite{scrR} to help the navigate web pages, the proposed web app seems to be structured in a way that may make this difficult. No mentions have yet been made to address making things more friendly for screen reader use. As a result, a subgroup of people would feel marginalized due to not being able to used the tarot reading service. 
    
    Other groups that may be affected would include those that have disabilities preventing them from being able to effectively use a keyboard. For example, some people have difficulties with fine motor control\cite{ins2}, which may make them unable to use a keyboard or press the wrong buttons. This group of individuals would also feel marginalized by the app. 
    
    The proposed web app consists of major accessibility issues that may also involve other groups of people besides those already discussed. Because of this, it should not be created unless proper efforts are put into making the project more accessible to a wider range of communities. These communities already face accessibility issues when trying to access services on the internet; the web app would only add to an already existing issue.

\section{Shifts in the Tarot Reading Market and it's Effects}
    If successful and launched to the general public, this project has the potential to take away clientele from human readers, similar to how technological alternatives to human work in general are projected to reduce the available jobs\cite{strack_carrasco_kolo_nouri_priddis_george_2021}. If the readings generated by the algorithm are felt to be a substitute in any way to human created readings, they may switch over to using the web app instead of embarking on their normal consumer behaviors. This would include readers who sell readings directly as well as those that post free readings on social media sites, using ad revenue as an income stream. 
    
    In some cases, technological alternatives may not have cascading effects to the general economy. This would be true if they still result in similar or increased levels of income and spending. However, in other cases, such as this one, this wouldn't be true. The web app would be free of cost to use, so it would lessen overall income for tarot readers.
    
    Faced with reduced incomes, tarot readers may spend less on other goods and services. Those that face reduced incomes as a result of reduced spending by tarot readers would also reduce their own spending, and the cycle would continue on wards. Overall, this would affect not only the tarot reading market, but also the market in general. This phenomenon is called the multiplier effect\cite{ganti_2022} and has been noticed during economic recessions such as the Great Recession of 2009 where multiple businesses and laborers lost part of their incomes as a result of others losing their incomes\cite{temple_zito_2012}. 
    
    Similarly, as a result, these people who would be firsthand or also indirectly affected by the reduced income would face lower levels of economic powers. Some may even be unable to afford as many necessity goods, which would be especially damaging socially as well. Since the effects of this free web app could cause negative effects on the general market, it would be very harmful to create.

\section{Users Having an Over-reliance on Tarot for Decision Making or Believing Them to be Truth Full}
    Although the app is said to be for entertainment purposes only, users may not take it as such. Instead, some users might believe the readings to be true every time they decide to use the service. For instance, let's suppose a user decides to consult the website for a love reading, with her relationship with her partner in mind. Now, let's say the generated reading tells her that her partner is cheating and intends to harm her. As a result, if she believes the reading to be completely true, she may start doubting her partner's actions. As a result, her relationship may eventually suffer due to trust issues. Although made up, this is a very real possibility if people began to put the tarot reading service the project proposes on a pedestal. 
    
    Additionally, some people have a tendency to over-rely on tarot. For example, Remy Ramirez would rely on her tarot deck to make decisions about her life for many years.\cite{ramirez_2021}. This would prevent those users to depend on logic and rationality to make choices. As a result, they may end up choosing options that may negatively affect their life in the long run. This would especially be the case if the readings are inaccurate in the first place. For example, imagine a user asks for a career reading with the choice of accepting or declining a recent job offer in the back of their mind. Let's say the reading they receive indirectly encourages them to not accept the job offer that may have provided extensive benefits throughout their life. This would have very major consequences.
    
    Because the web app would have the potential to cause negative significant effects in users' lives if used incorrectly, it is too dangerous to be released. Even if the developer's intentions are not to cause harm, the app could cause a significant amount of it regardless.


\section{User Testing Biases}
The app will supposedly be tested only on the developer's social media followers. As a result, it would suffer greatly from sampling bias. Sampling bias refers to "When some members of the target population are less likely to be included in the study."\cite{technology}. The way the user testing is designed would exclude virtually everyone except the developers' followers, which may result in responses to the testing that are not well-represented of the greater population of users that seek online automated tarot readings. Additionally, the user testing group proposed may respond in a biased way as they may have be more likely to respond positively to the app, even despite issues. This would mainly be because the fact that the testers would be followers may result in them rating the app more positively.

Moreover, the testing would also suffer from non response bias. Non response bias refers to deviations from the true population opinions in comparison to the sample that responded to the survey \cite{bose_2001}. Perhaps those who strongly feel in a positive or negative way about online automated tarot readers would choose to volunteer for user testing. For example, let's say that people who feel more positive than the mean decide to volunteer for the testing. They may be more forgiving of the apps shortcomings, such as those when considering coherence and readability. As a result, they may provide opinions that would result the developer in making less overall improvements to the service. The app, as a result, would not achieve it's optimal usefulness. Additionally, if users that have a strong liking to tarot already decide to volunteer, they would likely rate the experience of the app to be more entertaining than it really is. This may result in the adventure aspect of the project to be less stimulating in comparison. 

These two forms of biases, which do not compromise all the user testing biases this project may suffer from, would ultimately cause the app to be created based upon opinions that would be non representative of the population at large that they are trying to serve. In fact, perhaps the general public would view the app to be non entertaining and the readings to be incoherent. As a result, this app may not be very useful by the general public (or at least the target audience). The project may falsely be labeled as a success when it is in fact a failure. Therefore, it should not be tested in this way.

\section{Consent}
There is much debate about whether it is ethical or not to read someone else's tarot cards without receiving their consent first \cite{holly_2021}. The someone else in this case may refer to celebrities or other people that a user may even know personally. Especially if the readings received are false, this issue can be even more significant. Either way, it is arguably ethically considered wrong to obtain information about others without their consent, through whichever means it is received. 

On one hand, let's suppose the readings are false. For example,let's say a user who was curious whether their crush thought well of them decided to consult the web app. The reading they receive, to their dismay, told them that the person thought negatively of them. Especially if the user believed the reading to be true, they may end up not making an effort to pursue this person. If we suppose these two had excellent compatibility and would have gotten into a rewarding and long term relationship, the reading in this case would have prevented that from happening. This would be bad not only for the user consulting the web app, but also the person she inquired about.

On the other hand, let's now suppose that the readings are somehow accurate. In this case, people who mean harm to others may decide to misuse them against others' consent. This would be very detrimental. 

The misuse of readings against those who did not consent to them, whether false or not, is significant. Therefore, this app shouldn't be set into motion.

\section{Misunderstandings About the Origin of Readings}
Some people view tarot readings to be completely random and thus insignificant in their content, while others see it as ways to communicate with divine beings\cite{guru_2021}. Certain users lie somewhere in the middle of this scale, believing them to be somewhat significant but not to be taken too seriously. The first types of users would typically be non religious or non spiritual. 

For those that view the content of the readings to lack significance, the app may be highly useful. These users would essentially use the service in order to pass time and be entertained while doing it. The app may actually benefit these groups of people at a disproportionate level, assigning them with more power over the other users. 

However, it can be especially harmful if people believe that the readings generated on the site have a divine origin. As a result, they may place more emphasis on the content of the readings. They may also be more likely to believe them to be truthful in the cases where they are false. These type of users would be disproportionately affected by the ethical considerations discussed in the "Users Having an Over-reliance on Tarot for Decision Making or Believing Them to be Truth Full" section of this paper. Spiritual and religious people specifically may be at greater risk. As a result, the app would have the power to negatively affect these people's lives in the long run in a greater proportion. 
Since the web app would affect religious and spiritual people especially and victimize them, it shouldn't be made. It would also give non spiritual non religious people an unfair advantage. 



\section{Conclusion and Potential Solutions}
Although there are some solutions that may be used to address certain issues, other considerations are impossible to combat. 

For example, for the issue regarding accessibility for people with low vision or blindness, the page could be structured in a way that is easy to navigate with the use of a screen reader.\cite{institute} Additionally, audio output for the dialogue from the character and reading itself may be available as well in addition to alternative texts.

A possible way to reduce the negative effects on the tarot reading and other markets, the app could be held back from being available to the general public. Alternatively, a price could also be attached to the service in order to prevent as many people from using it. Even still, depending on how low the price is, the markets may be affected either way.

There are many statistical practices that can help reduce biases during user testing. This may include somehow randomly selecting participants from a more generalized sample (rather than just social media followers).\cite{hayes_2022}. However, since the developer wouldn't have the means to force people into testing his app, non response bias would ultimately end up affecting the results of the testing anyway. 

For the issue regarding an over reliance on false readings, not many solutions are possible. The developer may decide to present the users with the disclaimer that the readings are meant for entertainment only. However, there is no guarantee that the users would listen to it. 

Similarly, a developer could also provide a disclaimer or dialogue that discourages users from using readings without receiving other people's consent first. However, there is still no way to ensure that the users would choose to abide with that. 

Lastly, the developer could potentially make it explicit that the generated readings are done through a machine learning algorithm and that no divine forces were intentionally involved. Still, some users may choose not to believe that and place religious or spiritual weight on the readings anyway. As a result, they may even turn to over rely on them, which has the significant negative implications discussed above.

Given the fact that not all ethical issues would be able to be prevented, this app should not be created. It may have significantly negative effects on people's lives that are irreversible. 

\printbibliography 

\end{document}